\usepackage[english,russian]{babel} 

\usepackage{cmap} 
\usepackage[T2A]{fontenc} 
\usepackage[utf8]{inputenc} 

\usepackage[14pt]{extsizes} 



\usepackage{geometry}
\geometry{left=30mm}
\geometry{right=15mm}
\geometry{top=20mm}
\geometry{bottom=20mm}
\setlength{\parindent}{1.5cm}

\newcommand{\centerchapter}[1]{\titleformat{\chapter}{\filcenter\bfseries\Large}{\thechapter}{20pt}{}\addcontentsline{toc}{chapter}{#1}\chapter*{#1}}
\newcommand{\averchapter}[1]{\titleformat{\chapter}{\large\bfseries\LARGE}{\thechapter}{20pt}{}\chapter{#1}}

\usepackage{titlesec}

\titleformat{\section}{\normalsize\bfseries}
{\thesection}{1em}{}

\titlespacing*{\chapter}{0pt}{-30pt}{8pt}
\titlespacing*{\section}{\parindent}{*4}{*4}
\titlespacing*{\subsection}{\parindent}{*4}{*4}

\usepackage{titlesec}
\titleformat{\chapter}{\LARGE\bfseries}{\thechapter}{20pt}{\LARGE\bfseries}
\titleformat{\section}{\Large\bfseries}{\thesection}{20pt}{\Large\bfseries}



\usepackage{setspace}
\onehalfspacing 


\frenchspacing
\usepackage{indentfirst}



\usepackage{listings}
\usepackage{xcolor}

\lstset{ %
	language=python,   					
	basicstyle=\small\sffamily,			
	numbers=left,						
	%numberstyle=,					
	stepnumber=1,						
	numbersep=5pt,						
	frame=single,						
	tabsize=4,							
	captionpos=t,						
	breaklines=true,					
	breakatwhitespace=true,				
	escapeinside={\#*}{*)},				
	backgroundcolor=\color{white},
}


% PDF
\usepackage[justification=centering]{caption} 
\usepackage[unicode,pdftex]{hyperref} 
\hypersetup{hidelinks}



% Подписи
\captionsetup{labelsep=endash} 

% Таблицы
\usepackage{threeparttable}

% Рисунки
\usepackage{caption}
\captionsetup[figure]{name={Рисунок}} 

\usepackage{float}
\usepackage[justification=centering]{caption}
\usepackage{pgfplots}
\pgfplotsset{compat=1.9}
\usetikzlibrary{datavisualization}
\usetikzlibrary{datavisualization.formats.functions}
\usepackage{graphicx}
\newcommand{\img}[3] {
    \begin{figure}[h]
        \center{\includegraphics[height=#1]{assets/img/#2}}
        \caption{#3}
        \label{img:#2}
    \end{figure}
}
\newcommand{\boximg}[3] {
    \begin{figure}[h]
        \center{\fbox{\includegraphics[height=#1]{assets/img/#2}}}
        \caption{#3}
        \label{img:#2}
    \end{figure}
}


% Формулы
\usepackage{amsmath}
\newcommand{\code}[1]{\texttt{#1}} 